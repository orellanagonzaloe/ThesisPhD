% \chapter{TO DO}
% \chaptermark{TO DO}

{\LARGE To Do}

\begin{itemize}
	\item Siglas: ATLAS, SM, ID, EM
	\item Mencionar
	\begin{itemize}
		\item electrones = positrones
		\item leptones: sin tau
		\item MET asociado a neutrinos y nueva fisica
	\end{itemize}
	\item Definiciones
	\begin{itemize}
		\item qué cosas van con itálica (trigger?)
		\item pile up, convención para escribirlo
		\item Run 1, 2
		\item prompt
		\item crack region
		\item barrel, endcap
		\item z0 y sigmad0
		\item prescale y rerun
		\item derivation
	\end{itemize}
\end{itemize}


\vspace{2cm}


{\LARGE Citas}

\begin{itemize}
	\item newt: T. Cornelissen et al.,Concepts, Design and Implementation of the ATLAS New Tracking (NEWT),ATL-SOFT-PUB-2007-007 (2007),url:http://cds.cern.ch/record/1020106
	\item Kalman: R. Frühwirth,Application of Kalman filtering to track and vertex fitting, Nucl. Instrum. Meth. A262(1987) 444.
	\item chi2: T. G. Cornelissen et al.,The global chi2 track fitter in ATLAS, J. Phys. Conf. Ser.119(2008) 032013
	\item gsf: ATLAS Collaboration,Improved electron reconstruction in ATLAS using the Gaussian Sum Filter-based model for bremsstrahlung, ATLAS-CONF-2012-047, 2012,url:https://cds.cern.ch/record/1449796
	\item Cacciari: M. Cacciari and G. P. Salam,Pileup subtraction using jet areas, Phys. Lett. B659(2008) 119, arXiv:0707.1378 [hep-ph].
	\item silicon: ATLAS Collaboration,Performance of the ATLAS Silicon Pattern Recognition Algorithm in Dataand Simulation at s=7TeV, ATLAS-CONF-2010-072 (2010),url:http://cds.cern.ch/record/1281363.
	\item trt: ATLAS Collaboration,Particle Identification Performance of the ATLAS Transition RadiationTracker, ATLAS-CONF-2011-128, 2011,url:https://cds.cern.ch/record/1383793.
	\item trimming: JHEP 02, 084 (2010), 0912.1342
	\item btag:  ATLAS b-jet identification performance and efficiency measurement with ttbar events in pp collisions at sqrt(s) 13 TeV, Eur. Phys. J. C 79 (2019) 970, arXiv:1907.05120, FTAG-2018-01 
	\item jeffrey: arXiv:0908.0130

\end{itemize}


\vspace{2cm}


{\LARGE Notas}

\newcommand{\commentNotaI}{\pdfcomment[color=Orchid1, hspace=25pt]{Nota 1: Ciertas magnitudes no tiene sentido ponerlas explícitamente ya que no aportan mucho. Aún así el texto no debería quedar vago...}} \commentNotaI

\newcommand{\commentNotaII}{\pdfcomment[color=Orchid1, hspace=25pt]{Nota 2: La expresión `trigger' se utiliza en diferentes contextos que no son lo mismo, para un solo algoritmo, para en conjunto de triggers, para todo el sistema en sí, etc. Capaz habría que aclarar esta `vaguedad' en la expresión.}} \commentNotaII

\newcommand{\commentNotaIII}{\pdfcomment[color=Orchid1, hspace=25pt]{Nota 3: Los gráficos colocados en general son `dummy'. Cuando esté la mayor parte de la tesis escrita pienso dedicarme a buscar la versión de mejor calidad o emprolijarlos yo mismo. Tal vez los haga de nuevo, con colores uniformes para toda la tesis ?}} \commentNotaIII

\vspace{2cm}


{\LARGE Dudas}

\begin{itemize}
	\item Fotones convertidos dejan 2 depósitos en el EM o 1? Figura? Al parecer dejan 2 pero están muy juntos al estar boosteados, no es que dejan 2 depósitos bien separados en el detector y esos 2 se unifican en un fotón. De todas formas me gustaría confirmarlo
	\item En nuestro análisis los taus están como jets? No hay veto?
	\item Usamos los objetos baseline para calcular MET? Al final NO. MET se calcula por default por la tool con todos los objetos presentes en el evento
	\item El pt de los muones se mide solo con su traza? o algo de su energia se deposita en el MS y con eso se puede deducir el pt?
	\item Si al pi0 lo reconstruimos como a un jet, por que en la figura 3.2 hablamos de los 2 depositos de energía que deja su decaimiento a 2 fotones?
	\item Cuestiones sobre la definicion de objetos prompt: \\
	No entiendo tu comentario, la identificación son criterios de calidad del objeto que ya se clasificó como electrón o fotón. \\
	En	el paper EGAM-2018-01 se utiliza la jerga prompt vs background para motivar la iden-
	tificación. Aquellos e/y que vienen de decaimientos prompt se depositan en el ECAL y
	los podemos considerar prompt, aquellos que vienen en otro tipo de decaimiento supongo
	que estarán contenidos dentro de los jets... \\
	Revisar mejor definicion de prompt
\end{itemize}

