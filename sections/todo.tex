% \chapter{TO DO}
% \chaptermark{TO DO}

{\LARGE To Do}

\begin{itemize}
	\item Siglas: ATLAS, SM, ID, EM
	\item Mencionar
	\begin{itemize}
		\item electrones = positrones
		\item leptones: sin tau
		\item MET asociado a neutrinos y nueva fisica
	\end{itemize}
	\item Definiciones
	\begin{itemize}
		\item pile up
		\item Run 1, 2
		\item prompt
		\item crack region
	\end{itemize}
\end{itemize}


\vspace{2cm}


{\LARGE Citas}

\begin{itemize}
	\item Kalman: R. Frühwirth,Application of Kalman filtering to track and vertex fitting, Nucl. Instrum. Meth. A262(1987) 444.
	\item Cacciari: M. Cacciari and G. P. Salam,Pileup subtraction using jet areas, Phys. Lett. B659(2008) 119, arXiv:0707.1378 [hep-ph].
	\item newt: T. Cornelissen et al.,Concepts, Design and Implementation of the ATLAS New Tracking (NEWT),ATL-SOFT-PUB-2007-007 (2007),url:http://cds.cern.ch/record/1020106
	\item silicon: ATLAS Collaboration,Performance of the ATLAS Silicon Pattern Recognition Algorithm in Dataand Simulation at s=7TeV, ATLAS-CONF-2010-072 (2010),url:http://cds.cern.ch/record/1281363.
	\item chi2: T. G. Cornelissen et al.,The global chi2 track fitter in ATLAS, J. Phys. Conf. Ser.119(2008) 032013
	\item gsf: ATLAS Collaboration,Improved electron reconstruction in ATLAS using the Gaussian Sum Filter-based model for bremsstrahlung, ATLAS-CONF-2012-047, 2012,url:https://cds.cern.ch/record/1449796
	\item trt: ATLAS Collaboration,Particle Identification Performance of the ATLAS Transition RadiationTracker, ATLAS-CONF-2011-128, 2011,url:https://cds.cern.ch/record/1383793.
\end{itemize}


\vspace{2cm}


{\LARGE Nota 1}

Ciertas magnitudes no tiene sentido ponerlas explícitamente ya que no aportan mucho. Aún así el texto no debería quedar vago...


\vspace{2cm}


{\LARGE Dudas}

\begin{itemize}
	\item Fotones convertidos dejan 2 depósitos en el EM o 1? Figura?
	\item En nuestro análisis los taus están como jets?
	\item Usamos los objetos baseline para calcular MET?
\end{itemize}

