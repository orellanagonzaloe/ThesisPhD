% \chapter{TO DO}
% \chaptermark{TO DO}

{\LARGE To Do}

\begin{itemize}
	\item Siglas: ATLAS, SM, ID, EM
	\item Mencionar
	\begin{itemize}
		\item electrones = positrones
		\item leptones: sin tau
		\item MET asociado a neutrinos y nueva fisica
	\end{itemize}
	\item Definiciones
	\begin{itemize}
		\item qué cosas van con itálica (trigger?)
		\item pile up, convención para escribirlo
		\item Run 1, 2
		\item prompt
		\item crack region
		\item barrel, endcap
		\item z0 y sigmad0
		\item prescale y rerun
		\item derivation
		\item PDF
		\item RoI
	\end{itemize}
\end{itemize}


\vspace{2cm}


{\LARGE Citas}

\begin{itemize}
	\item newt: T. Cornelissen et al.,Concepts, Design and Implementation of the ATLAS New Tracking (NEWT),ATL-SOFT-PUB-2007-007 (2007),url:http://cds.cern.ch/record/1020106
	\item Kalman: R. Frühwirth,Application of Kalman filtering to track and vertex fitting, Nucl. Instrum. Meth. A262(1987) 444.
	\item chi2: T. G. Cornelissen et al.,The global chi2 track fitter in ATLAS, J. Phys. Conf. Ser.119(2008) 032013
	\item gsf: ATLAS Collaboration,Improved electron reconstruction in ATLAS using the Gaussian Sum Filter-based model for bremsstrahlung, ATLAS-CONF-2012-047, 2012,url:https://cds.cern.ch/record/1449796
	\item Cacciari: M. Cacciari and G. P. Salam,Pileup subtraction using jet areas, Phys. Lett. B659(2008) 119, arXiv:0707.1378 [hep-ph].
	\item silicon: ATLAS Collaboration,Performance of the ATLAS Silicon Pattern Recognition Algorithm in Dataand Simulation at s=7TeV, ATLAS-CONF-2010-072 (2010),url:http://cds.cern.ch/record/1281363.
	\item trt: ATLAS Collaboration,Particle Identification Performance of the ATLAS Transition RadiationTracker, ATLAS-CONF-2011-128, 2011,url:https://cds.cern.ch/record/1383793.
	\item trimming: JHEP 02, 084 (2010), 0912.1342
	\item btag:  ATLAS b-jet identification performance and efficiency measurement with ttbar events in pp collisions at sqrt(s) 13 TeV, Eur. Phys. J. C 79 (2019) 970, arXiv:1907.05120, FTAG-2018-01 
	\item jeffrey: arXiv:0908.0130
	\item CMS: S. Chatrchyan et al. The CMS Experiment at the CERN LHC. JINST, 3:S08004, 2008.
	\item ALICE: K. Aamodt et al. 3:S08002, 2008. The ALICE experiment at the CERN LHC. JINST,
	\item LHCb: A. Augusto Alves, Jr. et al. The LHCb Detector at the LHC. JINST, 3:S08005, 2008.
	\item magnet: https://cds.cern.ch/record/409763
	\item level1: R. Achenbach et al. The ATLAS level-1 calorimeter trigger. JINST, 3:P03001, 2008.
	\item grid: CERN. The worldwide lhc computing grid.
	\item analysistools: B Lenzi. The Physics Analysis Tools project for the ATLAS experiment.Technical Report ATL-SOFT-PROC-2009-006, CERN, Geneva, Oct 2009.
	\item athena: P Calafiura, W Lavrijsen, C Leggett, M Marino, and D Quarrie. The Athena 	Control Framework in Production, New Developments and Lessons Learned. 	2005.
	\item root: R. Brun and F. Rademakers. ROOT: An object oriented data analysis framework. Nucl. Instrum. Meth., A389:81–86, 1997.
	\item dis: V.N. Gribov and L.N. Lipatov. Deep inelastic scattering e p scattering in perturbation theory. Sov. J. Nucl. Phys., 15:438 (1972).
	\item lipatov\_parton: L.N. Lipatov. The parton model and perturbation theory. Sov. J. Nucl. Phys., 20:94
	(1975).
	\item qcd\_collider: R.K. Ellis, W.J. Stirling, and B.R. Webber. QCD and collider physics. Cambridge
	monographs on particle physics, nuclear physics, and cosmology. Cambridge Univer-
	sity Press, 2003.
	\item feynman: R.P. Feynman. Very high-energy collisions of hadrons. Phys. Rev. Lett., 23:1415–
	1417 (1969).
	\item bjorken: J.D. Bjorken and E.A. Paschos. Inelastic electron-proton and $\gamma$-proton scattering
	and the structure of the nucleon. Phys. Rev., 185:1975–1982 (1969).
	\item cteq: James Botts, Jorge G. Morfin, Joseph F. Owens, Jianwei Qiu, Wu-Ki Tung,
	and Harry Weerts. Cteq parton distributions and flavor dependence of sea
	quarks. Physics Letters B, 304(1-2):159–166, Apr 1993.
	\item mstw1: A. D. Martin, W. J. Stirling, R. S. Thorne, and G. Watt. Parton distributions
	for the lhc. The European Physical Journal C, 63(2):189–285, Jul 2009.
	\item mstw2: A. D. Martin, W. J. Stirling, R. S. Thorne, and G. Watt. Uncertainties on α s
	in global pdf analyses and implications for predicted hadronic cross sections.
	The European Physical Journal C, 64(4):653–680, Oct 2009.
	\item mstw3: A. D. Martin, W. J. Stirling, R. S. Thorne, and G. Watt. Heavy-quark mass
	dependence in global pdf analyses and 3- and 4-flavour parton distributions.
	The European Physical Journal C, 70(1-2):51–72, Oct 2010.
	\item nnpdf:  Carrazza S. Deans C. Del Debbio L. Forte S. Guffanti A. Hartland N. Latorre
	J. Rojo J. et al. Ball R., Bertone V. Parton distributions with lhc data. Nuclear
	Physics B, 867(2):244–289, Feb 2013.
	\item hierarchy:  S. Weinberg, Phys. Rev. D 13, 974 (1976), Phys. Rev. D 19, 1277 (1979); E. Gildener, Phys. Rev. D 14, 1667 (1976); L. Susskind, Phys. Rev. D 20, 2619 (1979); G. ’t Hooft, in Recent developments in gauge theories, Proceedings of the NATO Advanced Summer Institute, Cargese 1979, (Plenum, 1980).
	\item martin: Stephen P. Martin. A Supersymmetry primer. 1997. [Adv. Ser. Direct. High Energy Phys.18,1(1998)].
	\item gmsb1: Michael Dine and Willy Fischler. A Phenomenological Model of Particle Phy-
	sics Based on Supersymmetry. Phys. Lett. B, 110:227, 1982.
	\item gmsb2: Luis Alvarez-Gaume, Mark Claudson, and Mark B. Wise. Low-Energy Su-
	persymmetry. Nucl. Phys. B, 207:96, 1982.
	\item gmsb3: Chiara R. Nappi and Burt A. Ovrut. Supersymmetric Extension of the SU(3)
	x SU(2) x U(1) Model. Phys. Lett. B, 113:175, 1982.
\end{itemize}


\vspace{2cm}


{\LARGE Notas}

\newcommand{\commentNotaI}{\pdfcomment[color=Orchid1, hspace=25pt]{Nota 1: Ciertas magnitudes no tiene sentido ponerlas explícitamente ya que no aportan mucho. Aún así el texto no debería quedar vago...}} \commentNotaI

\newcommand{\commentNotaII}{\pdfcomment[color=Orchid1, hspace=25pt]{Nota 2: La expresión `trigger' se utiliza en diferentes contextos que no son lo mismo, para un solo algoritmo, para en conjunto de triggers, para todo el sistema en sí, etc. Capaz habría que aclarar esta `vaguedad' en la expresión.}} \commentNotaII

\newcommand{\commentNotaIII}{\pdfcomment[color=Orchid1, hspace=25pt]{Nota 3: Los gráficos colocados en general son `dummy'. Cuando esté la mayor parte de la tesis escrita pienso dedicarme a buscar la versión de mejor calidad o emprolijarlos yo mismo. Tal vez los haga de nuevo, con colores uniformes para toda la tesis ?}} \commentNotaIII

\newcommand{\commentNotaIV}{\pdfcomment[color=Orchid1, hspace=25pt]{Nota 4: Hay algunas dimensiones que se pueden omitir, tal vez esta sea una de ellas...}} \commentNotaIV

\vspace{2cm}


{\LARGE Dudas}

\begin{itemize}
	\item Fotones convertidos dejan 2 depósitos en el EM o 1? Figura? Al parecer dejan 2 pero están muy juntos al estar boosteados, no es que dejan 2 depósitos bien separados en el detector y esos 2 se unifican en un fotón. De todas formas me gustaría confirmarlo
	\item En nuestro análisis los taus están como jets? No hay veto?
	\item Usamos los objetos baseline para calcular MET? Al final NO. MET se calcula por default por la tool con todos los objetos presentes en el evento
	\item El pt de los muones se mide solo con su traza? o algo de su energia se deposita en el MS y con eso se puede deducir el pt?
	\item Si al pi0 lo reconstruimos como a un jet, por que en la figura 3.2 hablamos de los 2 depositos de energía que deja su decaimiento a 2 fotones?
	\item Cuestiones sobre la definicion de objetos prompt: \\
	HW: No entiendo tu comentario, la identificación son criterios de calidad del objeto que ya se clasificó como electrón o fotón. \\
	GO: En	el paper EGAM-2018-01 se utiliza la jerga prompt vs background para motivar la iden-
	tificación. Aquellos e/y que vienen de decaimientos prompt se depositan en el ECAL y
	los podemos considerar prompt, aquellos que vienen en otro tipo de decaimiento supongo
	que estarán contenidos dentro de los jets... \\
	Revisar mejor definición de prompt
	\item Hay muones que no lleguen al MS? los de bajo pT? si es así, agregarlos a los CT
\end{itemize}

