

{To Do}
electrones = positrones % Done!
leptones: sin tau o que el tau hadroniza % ya fue, no va
MET asociado a neutrinos y nueva fisica % Done
NLO % Done
Histfitter, hitfactory % Done
ARREGLAR TABLAS DE SISTEMATICOS % Done

{Cosas mal}
online/offline se usa antes de definirlo. aunque sea capaz lo pueda poner siempre con italica
cteq y las demas pdf citadas en theory y lhc

{Convenciones gramaticales}
Higgs siempre mayuscula
Palabras a revisar si van con mayuscula: lagrangiano, gaussiana
Siglas en los titulos?
La negrita para enfatizar conceptos, no la tengo bien definida todavia
footnote pegado a la palabra? de color seguros

{Italica}
Se usa para las palabras extranjeras. Si se usan mucho o son comunes en fisica (como spin), se pone en intalica la primera vez, y despues siempre italica. Si se usa dos o tres veces, siempre va en italica.
Lista de palabras:
Fake factors
Shower shapes

{Comillas}
Palabras en español que son parte de la jerga.
Lista de palabras:
jfakes: reales y falsos

{Dudas sobre formato}
jfake: region A, B, C y D


{Terminología a corregir}
Simulaciones de MC o simulaciones MC
Met femenino o masculino o nada?

{Posibles preguntas del jurado}
jfakes: por que los FF dependen de esas variables?
jfakes/efakes: el producto final de la CS, no esta claro como se hace...
por que datos y señal no tienen sistematicos: la comparacion es con los MC, si se lo ponemos a ambos es incluirla dos veces
sistematicos: por que la carga del muon es sist, y no la del electron
anti-kt: genera conos?
EWK: por que tth y ttg van juntos?
interpolacion: con que se hizo
estimador de Bayes con el método de Jeffrey ?
el HEC tiene dos ruedas, en que se diferencian?
agregar fondo higgs al analisis de ewk?
estudiar aceptancia y eficiencia
leer sobre taus
mirar lo del pile up, PRW



% {\LARGE Notas}

% \newcommand{\commentNotaI}{\pdfcomment[color=Orchid1, hspace=25pt]{Nota 1: Ciertas magnitudes no tiene sentido ponerlas explícitamente ya que no aportan mucho. Aún así el texto no debería quedar vago...}} \commentNotaI

% \newcommand{\commentNotaII}{\pdfcomment[color=Orchid1, hspace=25pt]{Nota 2: La expresión `trigger' se utiliza en diferentes contextos que no son lo mismo, para un solo algoritmo, para en conjunto de triggers, para todo el sistema en sí, etc. Capaz habría que aclarar esta `vaguedad' en la expresión.}} \commentNotaII

% \newcommand{\commentNotaIII}{\pdfcomment[color=Orchid1, hspace=25pt]{Nota 3: Los gráficos colocados en general son `dummy'. Cuando esté la mayor parte de la tesis escrita pienso dedicarme a buscar la versión de mejor calidad o emprolijarlos yo mismo. Tal vez los haga de nuevo, con colores uniformes para toda la tesis ?}} \commentNotaIII

% \newcommand{\commentNotaIV}{\pdfcomment[color=Orchid1, hspace=25pt]{Nota 4: Hay algunas dimensiones que se pueden omitir, tal vez esta sea una de ellas...}} \commentNotaIV

% \vspace{2cm}


{\LARGE Dudas}

\begin{itemize}
	\item Fotones convertidos dejan 2 depósitos en el EM o 1? Figura? Al parecer dejan 2 pero están muy juntos al estar boosteados, no es que dejan 2 depósitos bien separados en el detector y esos 2 se unifican en un fotón. De todas formas me gustaría confirmarlo
	\item En nuestro análisis los taus están como jets? No hay veto?
	\item Usamos los objetos baseline para calcular MET? Al final NO. MET se calcula por default por la tool con todos los objetos presentes en el evento
	\item El pt de los muones se mide solo con su traza? o algo de su energia se deposita en el MS y con eso se puede deducir el pt?
	\item Si al pi0 lo reconstruimos como a un jet, por que en la figura 3.2 hablamos de los 2 depositos de energía que deja su decaimiento a 2 fotones?

	pions are the most abundant particle produced in pp collisions at the Large Hadron Collider
(LHC), it is essential to characterize the calorimeter response to both charged ($\pi^{\pm}$) and neutral ($\pi^{0}$) pions. Neutral pions decay promptly to photon pairs with compact showers that are mostly captured by the
ATLAS electromagnetic calorimeter, while charged pions have more irregular showers that often require
the dense material in the ATLAS hadronic calorimeter to be stopped

	\item Cuestiones sobre la definicion de objetos prompt: \\
	HW: No entiendo tu comentario, la identificación son criterios de calidad del objeto que ya se clasificó como electrón o fotón. \\
	GO: En	el paper EGAM-2018-01 se utiliza la jerga prompt vs background para motivar la iden-
	tificación. Aquellos e/y que vienen de decaimientos prompt se depositan en el ECAL y
	los podemos considerar prompt, aquellos que vienen en otro tipo de decaimiento supongo
	que estarán contenidos dentro de los jets... \\
	Revisar mejor definición de prompt
	\item Hay muones que no lleguen al MS? los de bajo pT? si es así, agregarlos a los CT
\end{itemize}

Jfake:
vale tambien?: N_a * N_b = N_c * N_d