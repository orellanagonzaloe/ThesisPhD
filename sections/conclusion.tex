\chapter{Conclusión}
% \addcontentsline{toc}{chapter}{Conclusión}
\chaptermark{Conclusión}

Seleccionar fotones con una alta eficiencia es clave en el proceso de toma de datos del experimento ATLAS. Las desafiantes condiciones del LHC Run-2 requieren una constante optimización y mejora de la selección de disparadores y las técnicas utilizadas para para mantener las tasas por debajo de los límites al tiempo que proporciona objetos de manera eficiente para el análisis físico. El sistema de Trigger ATLAS ha demostrado funcionar cumpliendo estos requisitos durante la toma de datos Run-2, llegando a valores superiores al 90\% en distintas regiones como se determinó en el presente trabajo, incluso, en las condiciones de alta luminosidad que son ya dos veces el valor de diseño.

Basado en datos de colisión protón-protón con $\sqrt{s}=13\ \tev$ correspondiente a una luminosidad integrada de 139 \ifb\ registrada por el detector ATLAS
en el LHC en Run-2, se ha realizado una búsqueda con un estado final de al menos un fotón aislado con alto momento transverso, jets y alto momento transverso faltante. Se definen tres regiones de señal, una con una predicción de 2.67 $\pm$ 0.75 eventos de fondo y 2 eventos observados, otra con 2.55 $\pm$ 0.64 eventos de fondo y sin eventos observados, y la última que predice 2.55 $\pm$ 0.44 eventos de fondo con 5 eventos observados.
Los resultados son compatibles con ningún exceso significativo de eventos sobre la estimación de fondo de SM. Los límites superiores de 95 \% CL dependientes del modelo se establecen en las posibles contribuciones de la nueva física en un escenario GGM con un neutralino NLSP que es una mezcla de higgsino y bino. Para la correspondiente producción de gluino, las masas se excluyen a valores de 2200 ~ \gev\ para la mayoría de las masas NLSP investigadas. Los límites superiores de 95 \% CL independientes del modelo se establecen en la sección transversal visible asociada de las contribuciones de la nueva física.


