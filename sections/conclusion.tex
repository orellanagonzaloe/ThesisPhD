\chapter{Conclusión}
% \addcontentsline{toc}{chapter}{Conclusión}
\chaptermark{Conclusión}

Los datos recolectados por el LHC durante todo su período de actividad han sido de una relevancia sorprendente para diversos análisis en el área de la física de partículas. Dadas sus condiciones de elevada energía de centro de masa, junto con la capacidad de producir un cuantioso número de eventos de colisión, resulta un escenario ideal para la realización de diferentes experimentos que exploren regiones del universo no estudiadas hasta el momento. El detector ATLAS es uno de los principales experimentos del LHC cuyo sofisticado diseño se describe en el presente trabajo. Con el mismo, se han realizado distintas búsqueda de nuevas partículas, entre las que se destaca el descubrimiento del bosón de Higgs en 2012. Una vez entendido y corroborado el mecanismo de Higgs dentro del Modelo Estándar, nuevos horizontes quedan por explorar en extensiones del modelo que permitan explicar incógnitas del universo y la naturaleza aun no entendidas. Uno de los modelos teóricos de mayor interés para su estudio en el área de la física de partículas durante los últimos años es Supersimetría. El mismo es una extensión del Modelo Estándar que propone la existencia de nuevas partículas, aún no observadas, y que se espera que tengan masas del orden del TeV. 


Para cualquier tipo de análisis realizado en el experimento, es clave un correcto y eficiente funcionamiento del detector. Las desafiantes condiciones del Run 2 del LHC requieren una constante optimización y mejora de la selección de eventos de colisión, junto con la reconstrucción de los objetos que se desprenden de la misma, siendo la de fotones de particular interés para este trabajo.
El sistema de Trigger de fotones de ATLAS ha demostrado funcionar cumpliendo estos requisitos, llegando a valores de eficiencia superiores al $90\%$ durante la toma de datos del Run 2. Estos valores fueron determinados en el presente trabajo, para distintos parámetros del detector e incluso en las condiciones de alta luminosidad, que superan dos veces el valor de diseño. Los resultados obtenidos son luego utilizados por toda la colaboración, en todos los estudios que involucran la selección de fotones online en distintos estados finales, incluyendo análisis con Higgs decayendo a dos fotones. A su vez, se presentaron los factores de escala obtenidos para dichas eficiencias, empleadas para la corrección de las simulaciones de MC, los cuales se determinaron con valores cercanos a la unidad, para los distintos parámetros que los caracterizan.

En el presente trabajo se realizó búsqueda de nueva física utilizando los datos de colisión protón-protón con $\sqrt{s}=13\,\tev$ correspondiente a una luminosidad integrada de $139\,\ifb$ registrada, tomadas por el detector ATLAS en el LHC durante el Run 2. La misma estaba motivada por un modelo de ruptura de supersimetría denominado General Gauge Mediation, con producción de gluinos y su subsecuente decaimiento a neutralinos como NSLP, para luego decaer a fotones, Higgs y gravitinos. El estado final experimental que caracteriza a este modelo es entonces de al menos un fotón energético, jets y un alto momento transverso faltante. El método empleado para realizar la búsqueda consistió en la definición de regiones abundantes en eventos de señal, y la estimación de los procesos del Modelo Estándar que podían emular un estado final similar. Posibles discrepancias entre los datos observados y la estimación de esos procesos significaría una evidencia de un fenómeno no contemplado en el SM. 
Se definieron tres regiones de señal diseñadas para cubrir el espacio de parámetros de los modelos estudiados, que estaban caracterizados por la masa del gluino y del neutralino más liviano. Las mismas resultaron con una predicción de $2.67 \pm 0.75$ eventos de fondo y $2$ eventos observados, otra con $2.55 \pm 0.64$ eventos de fondo y sin eventos observados, y la última que predice $2.55 \pm 0.44$ eventos de fondo con $5$ eventos observados.
Los resultados son compatibles entonces con la estimación de fondos del SM. Los límites superiores de $95\%$ CL dependientes del modelo se establecen en las posibles contribuciones de la nueva física en un escenario GGM con un neutralino NLSP que es una mezcla de higgsino y bino. Para la correspondiente producción de gluino, las masas se excluyen a valores de hasta \magn{2.3}{TeV} para la mayoría de las masas NLSP investigadas. Los límites superiores de $95\%$ CL independientes del modelo se establecen en el número de eventos de nueva física, llegando a ser de $4.73$, $3$ y $7.55$ para cada región de señal. Dichos límites se establecen de forma equivalente para la sección eficaz visible asociada de las contribuciones de la nueva física, llegando a ser $0.034\,\ifb$, $0.022\,\ifb$ y $0.054\,\ifb$ respectivamente. 

Finalmente se presentó el diseño completo de una estrategia para una búsqueda de nueva física motivada por un modelo supersimétrico, cuyo estado final contenía al menos un fotón energético, jets y un alto momento transverso faltante, con producción electrodébil de gauginos. La metodología empleada diseña regiones sensibles a dicho modelo junto con su respectiva estimación de fondos del SM. La sensibilidad de descubrimiento del análisis se encuentra para neutralinos con masa cercana a los \magn{750}{GeV}, y establece límites esperados para las masas de los mismos del orden de los \magn{1.2}{TeV}.