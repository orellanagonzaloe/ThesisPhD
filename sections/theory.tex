\chapter{Modelo Estándar y Supersimetría}
% \addcontentsline{toc}{chapter}{Modelo Estándar y Supersimetría}
\chaptermark{Modelo Estándar y Supersimetría}

\section{Modelo estándar de la física de partículas}

El Modelo Estándar de la física de partículas (SM, por sus siglas en inglés) es la teoría que describe y clasifica a las partículas elementales de la naturaleza, junto con tres de las cuatro interacciones fundamentales conocidas hasta el momento. El mismo fue formulado en la década de los 70, a partir de varios trabajos científicos realizados durante la segunda mitad de ese siglo, entre los que se encuentra principalmente los realizados por Glasgow, Salam, Weinberg, Brout, Englert, Higgs. Para ese momento, el SM incorporaba a todas las partículas conocidas y predecía la existencia de otras adicionales, lo que motivó al desarrollo de nuevos detectores para realizar dichas búsquedas. El descubrimiento de nuevas partículas e interacciones predichas por el SM, junto con la medida de precisión de distintos parámetros del mismo, han convertido al SM en una teoría ampliamente aceptada por toda la comunidad científica, con una formulación matemática que sirve a su vez para nuevas futuras teorías.

\subsection{Partículas y clasificación del SM}

Las partículas en el SM se clasifican a primer orden entre bosones y fermiones. Los primeros son los mediadores de las interacciones entre las distintas partículas del modelo. El primero de ellos es el fotón ($\gamma$), mediador de la interacción electromagnética, que afecta a las partículas que tienen carga eléctrica. No hay una fecha exacta del descubrimiento del mismo, pero se puede entender a la descripción del efecto fotoeléctrico por parte del Albert Einstein, como la primera formulación con objetos discretos de esta interacción. A su vez están los bosones $W$ y $Z$, asociados a la interacción débil y gobiernan los intercambios de 'sabor' de las partículas. Descubiertos de forma propia (no sólo su interacción) en 1983 en el Super Proton Synchrotron del CERN. Por otro lado se encuentran los gluones, mediadores de la interacción fuerte de aquellas partículas con carga de 'color'. Su primera observación experimental se realizó en 1978 en el detector PLUTO del colisionador electrón-positrón DORIS del DESY. Finalmente está el bosón de Higgs, partícula asociada al mecanismo Brout-Englert-Higgs que describe el rompimiento espontáneo de simetría electrodébil, asociado a la generación de masas de todas las partículas que componen al SM. El mismo fue descubierto en el 2012 por los experimentos ATLAS y CMS del CERN. Cabe mencionar que la bien conocida interacción gravitatoria no es incluida en el SM debido a las contradicciones que aparecen al querer combinarla con la teoría de la Relatividad General. La partícula hipotética de spin 2 que gobernaría a esta interacción se denomina gravitón.

Los fermiones están asociados a las partículas interactuantes o que forman la materia (aunque no necesariamente tengan masa), esto se debe a que al obedecer la estadística de Fermi-Dirac, no es posible encontrarlos simultáneamente en un mismo estado cuántico y por ende tienden a formar estructuras. A su vez, estos se clasifican en leptones y quarks. Los primeros son aquellos que no tienen carga de color y por ende no interactúan fuertemente. Existen seis leptones: electrón, neutrino electrónico, muón, neutrino muónico tau y neutrino tauónico, los cuales es común agruparlos en generaciones, que son pares de partículas que exhiben propiedades similares. Todos ellos pueden interactuar débilmente, y salvo por los neutrinos, también electromagnéticamente. Los quarks, en cambio, son los fermiones con carga de color, y por ende los que pueden interactuar fuertemente. También existen seis quarks que se agrupan en tres generaciones: up, down, charm, strange, top y bottom. Debido al efecto del confinamiento de color, los quarks nunca pueden ser observados independientemente en la naturaleza, sino que se los observa en estados ligados sin color, denominados hadrones. Cuando el hadrón se forma de un quark-antiquark se los llama mesones, y cuando es un conjunto de tres quarks se los llama bariones. Los bariones más conocidos son los protones y neutrones, compuesto por quarks de valencia $uud$ y $udd$ respectivamente, donde cada quarks toma uno de los tres posibles colores. Toda la materia ordinaria observada se compone de electrones y quarks up y down.

\tosolve{agregar citas}

\tosolve{agregar imagen de las partículas del SM}

\tosolve{agregar tabla con partículas y propiedades (carga, masa, interacción)}

\tosolve{agregar tabla con parámetros?}

\subsection{Breve descripción matemática del SM}

El SM se formula como una teoría cuántica de campos, en general considerada efectiva ya que por ejemplo describe los fenómenos en una escala donde la gravedad no tiene mucha injerencia. A su vez se compone de teorías de gauge en las que a partir de imponer simetrías en el lagrangiano, no solo están asociados cantidades conservadas como bien enuncia el Teorema de Emily Noether, sino que también implica la existencia de de interacciones mediadas por bosones de gauge. 

Hasta la actualidad todos los experimentos demuestran que con tres simetrías es necesario y suficiente para describir las interacciones conocidas. Estas simetrías otorgan a las distintas partículas respectivas cargas, que vienen a representar etiquetas que se les puede dar a las mismas, y que el conjunto de ellas describe en su totalidad las propiedades de cada una.

El grupo de simetría del SM se define como:

\begin{equation}
	\mathcal{G}_{\text{SM}} = SU(3)_C \otimes SU(2)_L \otimes U(1)_Y
\end{equation}

\tosolve{mencionar subindices}
La primer simetría a mencionar es la U(1), relacionada con la interacción electromagnética. El bosón de gauge requerido para mantener la invarianza se denomina $B_{\mu}$. El  índice $\mu$ está presente debido a que $B_{\mu}$ debe transformarse bajo rotaciones espaciales de la misma forma que la derivada tradicional, garantizando así que la partícula tenga spin 1.
A su vez, todas las partículas deben tener una segunda invarianza denominada SU(2) electrodébil. Los bosones de gauge asociados se denominan $W_{\mu}^{i}$. El índice $i$ representa cada uno de los tres bosones de spin 1  asociados a los tres generadores de las transformaciones SU(2).
Finalmente, la última simetría requerida es la SU(3). Los bosones de gauge asociados se denominan $G_{\mu}^{a}$. El índice a representa cada uno de los ocho bosones de spin 1  asociados a los tres generadores de las transformaciones SU(3). Estos bosones son los gluones, y la teoría que los describe es la cromodinámica cuántica (QCD).

Por su parte, los fermiones se describen mediante campos que definen estados dentro del espacio formado por las distintas simetrías. La simetría SU(2) es análoga al spin, partículas con spin 0 son singletes, con spin 1/2 forman dobletes y con spin 1 forman tripletes. En el caso de SU(2) electrodébil los fermiones izquierdos forman dobletes, y los derechos forman singletes:

\begin{equation}
	f_L = 
	\begin{pmatrix}
	\nu_{e} \\
	e^{-} \\
	\end{pmatrix}_{L},
	\begin{pmatrix}
	u_\alpha \\
	d_\alpha \\
	\end{pmatrix}_{L}
	\quad
	f_R = e^{-}_{R},u_{R\alpha},d_{R\alpha}
\end{equation}

Esta distinción entre izquierdos y derechos, es lo que hace que aparezca la violación de paridad electrodébil de forma natural en la teoría.

El índice que aparece en los estados de los quarks, $\alpha$, es para describir cómo los mismos se transforman en el espacio SU(3) de la misma forma que en SU(2). En SU(3) la representación básica son tripletes cuyas componentes representan los tres estados de color posible (denominados en general $r$, $g$ y $b$). Los quarks forman tripletes producto de la combinación de esos estados, mientras que los leptones forman un singlete sin color y por ello no requieren de este índice.

Con esto en mente, el lagrangiano se comienza a construir a partir del lagrangiano de la partícula libre, pero reemplazando la derivada ordinaria con la covariante, que con las simetrías consideradas, queda de la siguiente forma:


\begin{equation}
D_{\mu} = \partial_{\mu} - i g_{1} \frac{Y}{2}B_{\mu} - i g_{2} \frac{\tau^{i}}{2}W_{\mu}^{i} - i g_{3} \frac{\lambda^{a}}{2}G_{\mu}^{a}
\end{equation}
%
donde $Y$, $\tau$ y $\lambda$ son los respectivos generadores de las transformaciones U(1), SU(2) y SU(3); y $g_1$, $g_2$ y $g_3$ son constantes que representan la intensidad de cada acoplamiento y deben medirse experimentalmente. Una convención para escribir las ecuaciones de forma más compacta es utilizada, en donde los términos de $D_{\mu}$ que actúen en los fermiones con una forma de matriz diferente se anulan. Por lo que los $W_{\mu}^{i}$ actuando sobre leptones derechos (singletes de SU(2)) se anulan, y los $G_{\mu}^{a}$ actuando sobre leptones (singletes de SU(3)) se anulan. Quedando así el término fermiónico del lagrangiano:

\begin{equation}
\mathcal{L}_{\text{ferm}} = \sum_{\text{fermiones}} \bar{f}i\gamma^{\mu}D_{\mu}f
\end{equation}

Al desglosar los distintos términos de este lagrangiano es posible relacionar algunos de ellos tanto con observaciones experimentales, como con predicciones de la teoría. Por ejemplo, al mirar solo los términos U(1) y SU(2), se puede obtener el lagrangiano asociado a la teoría electrodébil, donde se obtienen las denominadas corrientes neutras y cargadas, que posteriormente dieron con el descubrimiento de los bosones $W$ y $Z$. A su vez es posible obtener relaciones entre las constantes del modelo, diferentes mecanismos para su medición. Algunas relaciones a mencionar son por ejemplo:

\begin{equation}
\begin{split}
g_2 & = e/\sin{\theta_w} \\
g_1 & = e/\cos{\theta_w}
\end{split}
\end{equation}
%
donde $\theta_w$ es un nuevo parámetro determinado experimentalmente, llamado ángulo de mezcla electrodébil \tosolve{poner valor?}. También la relación:

\begin{equation}
Q = T_3 + \frac{Y_W}{2} 
\end{equation}
%
donde $Q$ es la carga eléctrica, $Y_W$ es el anterior mencionado generador de U(1) que en este contexto se denomina hipercarga débil y $T_3$ es la tercer componente del isospin débil, que toma valores $1/2$ o $-1/2$ dependiendo del estado SU(2) del doblete, o $0$ si es un singlete.

De la misma forma se puede obtener el lagrangiano asociado a QCD mirando solo los términos SU(3). Aún así no es posible sacar conclusiones de la misma forma que para la teoría electrodébil, debido al confinamiento de los quarks y gluones, que no permiten observarlos de forma aislada en la naturaleza. En la sección \ref{sec:qcd} se describe algunos detalles de QCD.

Por último cabe mencionar que en ningún momento se hizo distinción alguna entre las familias de leptones, por lo que es posible reemplazar en cualquier momento al electrón por el muón y lo mismo para su neutrino, y las ecuaciones siguen valiendo de la misma forma. Esto se denomina universalidad leptónica, y es una propiedad que se ha observado a lo largo de los años en diferentes experimentos \tosolve{mencionar posibles violaciones?}.


\subsection{Mecanismo de Higgs}

La formulación hasta ahora descripta no incluye en ningún momento las masas de ninguna de las partículas. Esto se debe a que al agregar términos de masa explícitos al lagrangiano, como por ejemplo $m\psi\hat{\psi}$ o $m_B^2 B^{\mu}B_{\mu}$, el mismo pierde la invarianza de SU(2). Si se incluyeran a las masas 'a mano' la teoría termina teniendo cantidades físicas infinitas (\tosolve{entender bien, renormalización?}). La forma de incluir masas a la teoría sin que estas sean nulas es mediante el mecanismo de Higgs. 

\tosolve{revisar la formalidad de toda esta sección} Para ello se asume que en el SM el universo está inmerso en un campo de spin 0, denominado campo de Higgs. El mismo es un doblete en el espacio SU(2) y tiene hipercarga no nula en U(1), pero es un singlete en el espacio de color. Es un caso similar al del campo electromagnético, pero en el caso deHiggs no se consideran fuentes del campo en esta instancia. Los bosones de gauge y los fermiones pueden interactuar con este campo, y en su presencia dejan de tener masa nula. Si bien el lagrangiano conserva la simetría SU(2) y U(1), el estado fundamental no, en lo que se denomina un rompimiento espontáneo de simetría.

La parte escalar del lagrangiano de Higgs está dada por:

\begin{equation}
\mathcal{L} = (D^{\mu}\phi)^{\dagger}(D^{\mu}\phi) - V(\phi)
\end{equation}
%
donde el $\phi$ es un campo escalar complejo en la representación de SU(2):

\begin{equation}
	\phi = 
	\begin{pmatrix}
	\phi^{+} \\
	\phi^{+} \\
	\end{pmatrix} = \frac{1}{\sqrt{2}}
	\begin{pmatrix}
	\phi_{1} + i\phi_{2} \\
	\phi_{3} + i\phi_{4} \\
	\end{pmatrix}
\end{equation}
%
con hipercarga U(1), $Y=+1$. La derivada covariante en este termino es similar a la descripta en pero sin el término de color, y con los mismos bosones de gauge de SU(2) y U(1). Esta simetría $U(1)_Y$ adicional es necesaria para que la teoría genere un boson de gauge no masivo.
$V(\phi)$ es el potencial de Higgs, que para garantizar la renormalización de la teoría e invarianza de SU(2) y U(1), requiere ser de la forma:

\begin{equation}
	V(\phi) = - \mu^{2}\phi^{\dagger}\phi + \lambda(\phi^{\dagger}\phi)^{2}
\end{equation}
%
donde $\lambda$ es un parámetro que debe ser mayor a $0$ para garantizar un mínimo del potencial, quedando el comportamiento determinado por el signo del otro parámetro, $\mu$. Para $\mu^2>0$ el campo genera un valor de expectación de vacío (VEV) no nulo que rompe espontáneamente la simetría. El potencial $V(\phi)$ toma la forma de un sombrero mexicano \tosolve{agregar imagen} y tiene infinitos números de estados degenerados con energía mínima que satisfacen $\phi^{\dagger}\phi = v^2/2$. De esos estados se elige arbitrariamente el estado:

\begin{equation}
	\left<\phi\right> = \frac{1}{\sqrt{2}}
	\begin{pmatrix}
	0 \\
	v \\
	\end{pmatrix}
\end{equation}

Debido a la conservación de la carga solo un campo escalar neutro puede adquirir VEV, por lo que $\phi^0$ se interpreta como la componente neutral del doblete, y por ende $Q(\phi)=0$. El electromagnetismo no se modifica por el campo escalar VEV y la ruptura de simetría se representa como:

\begin{equation}
SU(2)_L \otimes U(1)_Y \rightarrow U(1)_Q
\end{equation}

Para estudiar el espectro de partículas, se estudia al campo alrededor del mínimo utilizando una expansión en la dirección radial:

\begin{equation}
	\phi = \frac{1}{\sqrt{2}}
	\begin{pmatrix}
	0 \\
	v + h \\
	\end{pmatrix}
\end{equation}

Al elegir una dirección particular tenemos tres simetrías globales rotas, y por el teorema de Goldstone, aparecen tres bosones escalares no masivos. Estos bosones de Goldstone son absorbidos por los bosones $W$ y $Z$, adquiriendo así su respectiva masa, mientras que la expansión en la dirección radial da la masa de la excitación $h$, que es la masa del boson de Higgs. De esta forma queda la masa de los bosones de la teoría de la forma:

\begin{equation}
\begin{split}
	M_{\gamma} & = 0 \\
	M_{W} & = \frac{g_2 v}{2} \\
	M_{Z} & = \frac{v}{2}\sqrt{g_1^2 + g_2^2} \\
	M_{h} & = \sqrt{2\lambda}v \\
\end{split}
\end{equation}

Este mecanismo también permite otorgar masas a los fermiones, incluyendo en el lagrangiano términos con acoplamiento del tipo Yukawa:

\begin{equation}
	\mathcal{L}_{\text{Yuk}} = g_f(\bar{\psi}_L\phi \psi_R) + \text{h.c.}
\end{equation}
%
siendo este ahora sí invariante de SU(2). La constante $g_f$ describe el acoplamiento entre el doblete de Higgs y los fermiones. Al hacer una expansión del campo como se hizo anteriormente, aparecen en el lagrangiano términos de masas fermiónicos que dan masa a los mismos de la forma:

\begin{equation}
m_f = \frac{g_f v}{\sqrt{2}}
\end{equation}

\tosolve{mención a las masas de quarks y neutrinos que hay en el Gordon Kane?)}

\tosolve{mencionar renormalizacion?}

\subsection{Comentarios adicionales sobre QCD}\label{sec:qcd}

Como se mencionó anteriormente, QCD \cite{qcdcollider} es la teoría de campos de gauge renormalizable
que describe la interacción fuerte entre quarks mediados por gluones. Los gluones son los objetos que generan las transiciones de un quark de color a otro. Las propiedades de los quarks en QCD son análogas de alguna forma a las del fotón en QED, con la distinción de que estos sí llevan carga (de color), y por ende pueden autointeractuar y además cambiar la carga  de color de los quarks (a diferencia de las partículas cargadas eléctricamente, que si bien pueden emitir o absorber un fotón, esto nunca cambia su carga). Esto se debe principalmente a la estructura no abeliana de su grupo de simetría.
Esto a su vez afecta a la constante de acoplamiento fuerte ($\alpha_s$) que termina dependiendo de la distancia de las cargas o la energía de la interacción (\textit{running coupling constant}). 

En QED, la polarización del vacío es inducida por los
pares virtuales $e^{+}e^{-}$, que apantallan (\textit{screening}) la carga eléctrica y resulta en la disminución del
acoplamiento con la distancia. Por el contrario, los gluones no sólo producen pares $q\bar{q}$ (que
causan un efecto análogo al de QED) sino que crean también pares de gluones adicionales,
que tienden a antiapantallar (\textit{anti-screening}) la carga aparente de color. El efecto neto es entonces que
el acoplamiento fuerte decrece con la energía y crece con la distancia \tosolve{sacado de la tesis de martin}. Esto da lugar al ya mencionado confinamiento de color, debido a que el potencial del campo de color aumenta linealmente con la distancia, y por lo tanto no se pueden observar quarks ni gluones libres en la naturaleza, solo observarlos en conjuntos sin color. Por otro lado, a pequeñas distancias o altas energías, se produce la libertad asintótica, donde la intensidad de
la interacción fuerte decrece, de tal forma que los quarks y gluones se comportan
esencialmente libres ($\alpha_s \ll 1$), posibilitando así un tratamiento perturbativo. 

En un colisionador de protones, los quarks y
gluones producidos altas energías sufren un proceso conocido como hadronización,
a medida que pierden energía, de manera que no se detectan quarks o gluones en el
detector, sino que unos objetos conocidos como jets, que son un chorro de hadrones
o una cascada de partículas, que forma un cono desde el quark/gluon inicial hasta el
calorímetro. \tosolve{sacado de la tesis de joaco}



\subsection{Colisiones protón-protón}

El LHC es principalmente un colisionador de protones, por lo que describir las interacciones que subyacen en la colisión misma no solo es importante para entender los fenómenos que se producen, sino también para poder generar simulaciones de dichos procesos con una elevada precisión. Si bien la colisión $pp$ es bastante eficaz a la hora de obtener colisiones de altas energías, la misma está gobernada principalmente por interacciones QCD que son complejas en su propia naturaleza a la hora de realizar una descripción teórica. Para ello se utiliza el modelo de partones, introducido por Feynman \cite{feynman} y Bjorken \cite{bjorken} a fines de los años 60. 

El modelo de partones propone que a altas energías los hadrones están compuesto por partículas puntuales denominadas partones, que vienen a representar los quarks de valencia y los quarks, antiquarks y gluones del mar presentes en el protón. Cada uno de los partones lleva entonces una fracción de la energía y momento del protón, que a priori son desconocidas. Esto representa un problema a la hora de medir secciones eficaces partónicas, $\sigma(qg\to qg)$, sumado a que tampoco es posible medir los productos finales de forma directa. En cambio, es posible medir una sección eficaz hadrónica, $\sigma(pp\to jj)$, entre los protones incidentes y los jets del estado final. Para realizar este pasaje se emplea el teorema de factorización \cite{ELLIS1978281}, que permite una separación sistemática
entre las interacciones de corta distancia (de los partones) y las interacciones de larga distancia (responsables del confinamiento de color y la formación de hadrones). El teorema establece que la sección eficaz de producción de cualquier proceso de QCD del tipo $A+B\to X$ puede ser expresada como:

\begin{equation}
	\label{eq:xs_fact}
	\sigma_{AB\to X} = \sum_{ij} \int dx_{a_i} dx_{b_j} f_{A/a_i}(x_{a_i}, \mu_{F}^2) f_{B/b_j}(x_{b_j}, \mu_{F}^2) \sigma_{a_i b_j \to X}(\mu_{F}^2, \mu_{R}^2)
\end{equation}
%

\tosolve{a partir de aca esta igual a la tesis de Martin, seguramente tenga que refrasear todo, pero el contenido me parecio bueno}
donde $x_i(x_j)$ es la fracción del momento del hadrón $A(B)$ que lleva el partón $a_i(b_j)$ y $\sigma_{a_i b_j \to X}$ es la sección eficaz de la interacción a nivel partónico, calcada a un dado orden en QCD perturbativo (pQCD) y una escala de renormalización $\mu_R$. La escala de renormalización es introducida
para absorber las divergencias ultravioletas que aparecen en los cálculos perturbativos más
allá de LO.

Las funciones $f_{h/n}(x_{n}, \mu_{F}^2)$, llamadas funciones de distribución partónica (PDFs), representan la probabilidad de encontrar un partón de tipo $n$ en el hadrón $h$ con una fracción de
momento $x_n$, dada una escala de factorización $\mu_{F}$. Esta escala es un parámetro arbitrario
introducido para tratar singularidades que aparecen en el régimen no perturbativo. Estas
divergencias son absorbidas, en forma similar a la renormalización, dentro de las funciones
de distribución partónicas a la escala $\mu_F$. Si bien las PDFs no pueden ser determinadas
perturbativamente, se puede predecir su dependencia con $Q^2$ por medio de las ecuaciones
de evolución DGLAP (Dokshitzer-Gribov-Lipatov-Altarelli-Parisi) \cite{dis,lipatovparton,altarelli-parisi}. De esta forma, la
medida experimental de su forma funcional a un dado $Q^2_0$ fijo permite obtener predicciones
de las PDFs para un amplio espectro de $Q^2$. En la presente tesis se consideran las predicciones teóricas a NLO utilizando las parametrizaciones CTEQ \cite{cteq}, MSTW \cite{mstw1, mstw2, mstw3} y NNPDF \cite{nnpdf}.

Luego de la interacción dura, cada partón del estado final comienza a radiar gluones,
perdiendo energía. Estos gluones fragmentan en pares $q\bar{q}$ y gluones adicionales, y así sucesivamente, creando una lluvia de partones, de cada vez más bajo \pt. Esto continúa hasta
que la energía es suficientemente baja y todos los partones se recombinan para formar
mesones y bariones, en lo que se conoce como hadronización. Las bajas transferencias de
energía involucradas en el proceso son tales que este no puede ser tratado perturbativamente. La dinámica de esta evolución es absorbida en funciones de fragmentación, que
representan la probabilidad de un partón de fragmentar en un determinado hadrón del
estado final. La sección eficaz $\sigma_{AB\to X}$ en la Ecuación \ref{eq:xs_fact} puede ser modificada entonces para
calcular el proceso $A + B \to C + X$:

\begin{equation}
	\sigma_{a_i b_j \to C+X} = \int dz_{C} D_{c_k}(z_C, \mu_{f}^2) \sigma_{a_i b_j \to c_k + X}(\mu_{F}^2, \mu_{R}^2)
\end{equation}

donde $C$ es un hadrón, $D_{c_k}$ es la función de fragmentación, que define la probabilidad
de que un partón $c_k$ fragmente en un hadrón $C$ con una fracción $z_C$ de su momento a la
escala de fragmentación (o factorización del estado final) $\mu_{f}$. Esta escala es introducida
de manera similar a $\mu_{f}$ para el estado inicial, a fin de remover las singularidades por
radiación colineal en el estado final.


\tosolve{aca joaco pone lo de underlying event, hard scatter, etc. me parece mejor ponerlo en la parte de detector, al final, en la seccion simulaciones de MC}












\subsection{Limitaciones del SM}

En la sección anterior se describió brevemente la mayoría de las propiedades del SM junto con sus predicciones. A pesar de ser una de las teorías más exitosas de la teoría cuántica de campos, naturalmente el modelo tiene un rango de validez. A lo largo de los años la frontera experimental de ha ido expandiendo, observando nuevos (y no tan nuevos) fenómenos que con la actual formulación del SM no puede explicar, principalmente en el rango de altas energías.

Una de las principales limitaciones del SM es la imposibilidad de incluir a la gravedad de la misma forma que incluye a las demás interacciones. No solo incluir al gravitón a la teoría no es suficiente para poder explicar las observaciones, sino que la matemática empleada en el SM es prácticamente incompatible con la formulación de la Relatividad General. Relacionado con esto está lo que se denomina el problema de jerarquía \cite{hierarchy}. Un problema de jerarquía en el contexto de física de partículas, se refiere a cuando alguno de los parámetros empleados por la teoría difiere en varios ordenes de magnitud, de otros parámetros equivalentes de la misma. Esto lleva a pensar que la formulación de esa teoría no sea del todo definitiva, y que en cambio está compensando ciertos defectos incluyéndolos en ese parámetro tan diferente. En el caso del SM, hay 17 órdenes de magnitud entre la escala electrodébil ($M_W\sim 10^{2}\ \gev$) y el escala de Planck ($M_P\sim 10^{19}\ \gev$), en donde los efectos de la gravedad cuántica comienzan a ser comparables con las demás interacciones.

Por otro lado, observaciones cosmológicas sostienen que el SM solo describe casi el 5\% de la materia 'visible', y que existe un 25\% de materia, denominada oscura, debido a que no se pudo observar mediante instrumentos que utilicen radiación electromagnética, pero sí a partir de sus efectos gravitatorios. El SM no provee una partícula candidata que logre cumplir todos los requisitos necesarios para ser materia oscura (eléctricamente nula y débilmente interactuante entre otras cosas). Tampoco explica la asimetría bariónica, ya que en el universo abunda la materia con respecto a la antimateria, y el SM no asume diferencias significativas entre ambos.

La observación de la oscilación de neutrinos da a entender que si bien los neutrinos tienen una masa muy pequeña, la misma no es nula, en contraposición con lo que formula el SM. Si bien hay varios mecanismos para concluir las mismas dentro del SM, no hay evidencia suficiente para saber cuál es la forma correcta, sumado a los nuevos desafíos teóricos que implica incluirla de estas formas (por ejemplo, existencia de nuevas partículas aun no observadas).

Por último cabe destacar que varios decaimientos o parámetros de la teoría han sido medidos con una elevada precisión, desviándose de los valores predichos por el SM \tosolve{muon g-2, b mesón decay, lepton universality, alguno mas, citas?}. Esto no necesariamente signifique un defecto de la teoría, pero muchas veces puede ser una motivación para la formulación de nuevas teorías.

\section{Supersimetría}

Retomando otro problema de jerarquía, el término de masa del Higgs recibe correcciones virtuales de cada partícula que se acople al campo de Higgs. Si el campo de Higgs acopla a un fermión $f$ con un término en el Lagrangiano de la forma $-\lambda_f H \bar{f} f$ entonces el diagrama de Feynman que aparece en la Figura \tosolve{agregar loop} genera una corrección:

\begin{equation}
	\Delta m_H^2 = - \frac{|\lambda_f|}{8 \pi^2}\Lambda_{\text{UV}}^2 + ...
\end{equation}

donde $\Lambda_{\text{UV}}$ es la escala de energía donde el SM deja de ser válido y nuevos fenómenos físicos pueden ser apreciables. Cualquier fermión del SM puede tomar el rol de $f$ pero la mayor corrección viene de parte del $top$ quark con un $\lambda_f\sim1$, y un factor 3 adicional por las cargas de color. Si $\Lambda_{\text{UV}}$ es del orden de $M_P$, las correcciones a la masa del Higgs son casi 30 órdenes de magnitud mayores a su valor medido. Si bien los demás bosones y fermiones del SM no tienen este problema de forma directa, al obtener la masa a partir de $\left<H\right>$ terminan siendo sensibles a esta escala de la misma forma.

\tosolve{mencionar alternativas para solucionar esto?} Una forma de solucionar esto es considerando que si existiera un complejo escalar pesado $S$, con masa $m_S$, que acopla al Higgs mediante un término del lagrangiano $-\lambda_S |H|^2|S|^2$, el diagrama de la Figura X genera una corrección:

\begin{equation}
	\Delta m_H^2 = - \frac{|\lambda_S|}{16 \pi^2}\left[\Lambda_{\text{UV}^2} - 2m_S^2 \ln{\Lambda_{\text{UV}}/m_S} + ... \right]
\end{equation}

Considerando la diferencia de signos entre el \textit{loop} fermiónico y bosónico, si cada fermión del SM estuviera acompañado por dos campos complejos con $\lambda_S = |\lambda_f|^2$ genera una cancelación automática de los términos. Esto motiva la inclusión de una nueva simetría a la teoría, entre fermiones y bosones, llamada Supersimetría (SUSY).

\subsection{Álgebra de SUSY}

Una transformación supersimétrica transforma un estado bosónico en un fermiónico y viceversa. El operador $Q$ que genera tal transformación tiene que ser un espinor anticonmutativo:

\begin{equation}
	Q\ket{\text{Bosón}} = \ket{\text{Fermión}} \quad Q\ket{\text{Fermión}} = \ket{\text{Bosón}}
\end{equation}

Los espinores son objetos complejos, por lo que $Q^{\dagger}$ es también un generador de la simetría. Como $Q$ y $Q^{\dagger}$ son operadores fermiónicos (tienen spin $1/2$), supersimetría es una simetría espaciotemporal, y deben cumplir las siguientes reglas de (anti)conmutación:

\begin{equation}
	\begin{split}	
		\{Q,Q^{\dagger}\} & = P^{\mu}\\
		\{Q,Q\} & = \{Q^{\dagger},Q^{\dagger}\} = 0\\
		[P^{\mu},Q] & = [P^{\mu},Q^{\dagger}] = 0\\
	\end{split}	
\end{equation}
%
donde $P^{\mu}$ es el cuadrivector generador de las traslaciones espaciotemporales (los índices sobre los operadores $Q$ y $Q^{\dagger}$ fueron suprimidos intencionalmente).


Los estados de partícula son representaciones irreducibles del álgebra de SUSY y se denominan supermultipletes. Cada uno contiene ambos estados bosónico y fermiónico, denominados supercompañeros. Como el operador $-P^2$ (cuyos autovalores son las masas) conmuta con los operadores $Q$ y $Q^{\dagger}$ y con los operadores de traslación y rotación, los supercompañeros dentro de un supermultiplete deben tener la misma masa. A su vez, como los operadores $Q$ y $Q^{\dagger}$ conmutan con los generadores de las transformaciones de gauge, los supercompañeros deben tener misma carga eléctrica, isospin débil y carga de color.

Cada supermultiplete debe contener igual número de grados de libertad fermiónica y bosónica, $n_F$ y $n_B$ respectivamente. Una forma posible de construir un supermultiplete con estas características es que tenga un solo fermión de Weyl con $n_F=2$ (dos estados de helicidad) y dos campos escalares reales cada uno con $n_B=1$ (los cuales se combinan en un campo escalar complejo). Este tipo de supermultipletes de denominan escalares o quirales. Otra posibilidad es combinar un boson vectorial de spin 1 (boson de gauge no masivo con dos estados de helicidad, $n_B=2$), con un fermión de Weyl no masivo de spin $1/2$ (con dos estados de helicidad, $n_F=2$). Por como se transformar los bosones de gauge, sus supercompañero fermiónicos deben tener las mimas propiedades de las transformaciones de gauge para sus componentes izquierdas y derechas. Este tipo de supermultiplete se denominan vectoriales o de gauge. Si incluimos a la gravedad, entonces el gravitón de spin 2 ($n_B=2$) tiene un supercompañero con spin $3/2$, y si es no masivo con dos estado de helicidad $n_F=2$. Hay otras posibilidades de partículas para generar supermultipletes, pero en general se terminan reduciendo a combinaciones de supermultipletes quirales y de gauge, excepto en teorías con supersimetrías adicionales. En nuestro caso inicial, la teoría se la denomina SUSY $N=1$, donde $N$ es el número de supersimetrías (o el número de conjuntos de operadores $Q$ y $Q^{\dagger}$). 

\subsection{El Modelo Estándar Supersimétrico Mínimo}

El Modelo Estándar Supersimétrico Mínimo (MSSM) es la extensión del SM que requiere incluir la mínima cantidad de partículas para completar los supermultipletes. Los supermultipletes solamente pueden ser escalares o vectoriales, pero solo los escalares pueden contener fermiones cuyas partes izquierdas y derechas se transformen distinto frente a los grupos de gauge, y como los fermiones del SM tienen esta propiedad se los incluye en este tipo de supermultiplete. Cada componente izquierda y derecha de los fermiones son separadas en fermiones de Weyl con diferentes transformaciones de gauge, por lo que cada una tiene su compañero complejo escalar, un bosón de spin 0. Los nombres de esto bosones son iguales al de su fermión correspondiente pero anteponiendo una 's' (por escalar en inglés), y lo mismo ocurre con su símbolo pero con una tilde. Por lo que tendríamos los \textit{selectrons} ($\tilde{e}_L$, $\tilde{e}_R$), \textit{smuons} ($\tilde{\mu}_L$, $\tilde{\mu}_R$), \textit{squarks} ($\tilde{q}_L$, $\tilde{q}_R$), etc (también vale \textit{sleptons} o \textit{sfermions} para el conjunto). Cabe mencionar que el índice en los \textit{sleptons} representa la helicidad del fermión correspondiente, y no su propia helicidad (que no tienen por ser de spin 0). En el caso de los neutrinos al ser siempre izquierdos sus \textit{sneutrinos} no necesitan subíndice salvo para indicar su sabor: $\tilde{\nu}_e$, $\tilde{\nu}_{\mu}$ o $\tilde{\nu}_{\tau}$. Las interacciones de los \textit{sfermions} son las mismas que su correspondiente fermión, por lo que los \textit{sfermions}$_L$ acoplan con el bosón $W$ pero los \textit{sfermions}$_R$ no.

El bosón de Higgs debe encontrarse en un supermultiplete escalar debido a que tiene spin 0, pero a su vez el MSSM requiere de la existencia de dos dobletes escalares complejos de Higgs, en lo que se denomina el modelo de doble doblete de Higgs (2HDM). A esos dobletes de SU(2)$_L$ con $Y=1/2$ e $Y=-1/2$ se los llama $H_u=(H_u^+, H_u^0)$ y $H_d=(H_d^0, H_d^-)$ respectivamente. El bosón escalar de Higgs del SM es una combinación lineal de las componentes de isospin débil neutras de ambos dobletes ($H_u^0$ y $H_d^0$). Los supercompañeros de los bosones se los denomina agregando 'ino' de su nombre, por lo que los supercompáneros de los dobletes de Higgs son los higgsinos, $\widetilde{H}_u=(\widetilde{H}_u^+, \widetilde{H}_u^0)$ y $\widetilde{H}_d=(\widetilde{H}_d^0, \widetilde{H}_d^-)$.

Por otro lado, los bosones de gauge del SM deben estar contenidos en un supermultiplete vectorial con sus respectivos supercompañeros denominados gauginos. El gluón tiene un supercompañero de spin $1/2$ denominado gluino ($\tilde{g}$). Por su parte, la simetría $SU(2)_L\times U(1)_Y$ asociada a los bosones de gauge $W^+$, $W^0$, $W^-$ y $B^0$ tienen sus supercompañeros $\widetilde{W}^+$, $\widetilde{W}^0$, $\widetilde{W}^-$ y $\tilde{B}^0$, llamados winos y bino. Luego de la ruptura de simetría electrodébil los estados de gauge $W^0$ y $B^0$ se mezclan en los estados de masa $Z^0$ y $\gamma$, y de la misma forma lo hacen los $\widetilde{W}^0$ y $\widetilde{B}^0$ para dar lugar al zino ($\widetilde{Z}^0$) y photino ($\tilde{\gamma}$).

En la Tabla \ref{mssm_particles} se resume todas las partículas requeridas por el MSSM, donde vale remarcar que ninguno de los supercompañeros del SM mencionados anteriormente ha sido observado experimentalmente hasta la fecha. Otro comentario de interés es que tanto el supermultiplete vectorial $H_d$ ($H_d^0$, $H_d^-$, $\widetilde{H}_d^0$, $\widetilde{H}_d^-$), como el de los \textit{sleptons} izquierdos ($\tilde{\nu}$, $\tilde{e}_L$, $\nu$, $e_L$) tienen los mismos números cuánticos. Esto podría llevar a pensar que no es necesario incluir u nuevo doblete de Higgs y en cambio utilizar el de los \textit{sleptons} izquierdos. si bien esto es posible, conlleva a diversos problemas fenomenológicos como violaciones en el número de leptones y necesidad de neutrinos del SM muy masivos, lo que motiva a descartar esto.


\begin{table} 

	\centering

	\begin{tabular}{ l c | c c c}

		\multicolumn{2}{l|}{Supermultipletes escalares} & Spin 0 & Spin $1/2$ & $SU(3)_C, SU(2)_L, U(1)_Y$ \\

		\hline

		\multirow{3}{*}{\textit{squarks}, quarks} & $Q$ & $(\tilde{u}_L\ \tilde{d}_L)$ & $(u_L\ d_L)$ & $(\textbf{3}, \textbf{2}, \frac{1}{6})$ \\
		 & $\bar{u}$ & $\tilde{u}_R^*$ & $u_R^{\dagger}$ & $(\bar{\textbf{3}}, \textbf{1}, -\frac{2}{3})$ \\
		 & $\bar{d}$ & $\tilde{d}_R^*$ & $d_R^{\dagger}$ & $(\bar{\textbf{3}}, \textbf{1}, \frac{1}{3})$ \\

		\hline

		\multirow{2}{*}{\textit{sleptons}, leptones} & $L$ & $(\tilde{\nu}\ \tilde{e}_L)$ & $(\nu\ e_L)$ & $(\textbf{1}, \textbf{2}, -\frac{1}{2})$ \\
		 & $\bar{e}$ & $\tilde{e}_R^*$ & $e_R^{\dagger}$ & $(\textbf{1}, \textbf{1}, 1)$ \\

		\hline

		\multirow{2}{*}{Higgs, \textit{higgsinos}} & $H_u$ & $(H_u^+\ H_u^0)$ & $(\widetilde{H}_u^+\ \widetilde{H}_u^0)$ & $(\textbf{1}, \textbf{2}, +\frac{1}{2})$ \\
		 & $H_d$ & $(H_d^0\ H_d^-)$ & $(\widetilde{H}_d^0\ \widetilde{H}_d^-)$ & $(\textbf{1}, \textbf{2}, -\frac{1}{2})$ \\

		\multicolumn{5}{c}{} \\

		\hline

		\multicolumn{2}{l|}{Supermultipletes vectoriales} & Spin $1/2$ & Spin 1 & $SU(3)_C, SU(2)_L, U(1)_Y$ \\

		\hline

		\multicolumn{2}{l|}{gluino, gluon} & $\tilde{g}$ & $g$ & $(\textbf{8}, \textbf{1}, 0)$ \\

		\hline

		\multicolumn{2}{l|}{winos, bosones $W$} & $\widetilde{W}^{\pm}\ \widetilde{W}^{0}$ & $W^{\pm}\ W^0$ & $(\textbf{1}, \textbf{3}, 0)$ \\

		\hline

		\multicolumn{2}{l|}{bino, bosón $B$} & $\widetilde{B}^0$ & $B^0$ & $(\textbf{1}, \textbf{1}, 0)$ \\

	\end{tabular}

	\caption{Espectro de partículas del MSSM. Solo una de las tres familias de fermiones y sfermions es mostrada. Por convención, las componentes derechas de los mismos aparecen como conjugados/adjuntos. También al lado del nombre de los supermultipletes escalares aparece el símbolo para representar al supermultiplete como un todo. La barra arriba de los fermiones y sfermions derechos es parte del nombre y no representa una conjugación \tosolve{entender esto y por que la negrita y la barra en los numeros}.}
	\label{mssm_particles}

\end{table}


\subsection{Ruptura de SUSY}

Como se mencionó anteriormente, la formulación presentada hasta ahora del MSSM propone la existencia de nuevas partículas cuyas masas son iguales a las masas de las partículas del SM. Por ejemplo, el \textit{selectron}$_L$ debería tener una masa de 511 keV, el photino y gluino masas nulas, y de la misma forma con todas las demás partículas del SM que no superan los 200 GeV. Este rango de energía ha sido ampliamente estudaido por distintos experimentos a lo largo de los años, y de existir partículas con esas masas debería haber sido una tarea fácil observarlas. Como este no ha sido el caso, se dice que supersimetría es una simetría débilmente rota. Se le define 'débilmente' ya que se necesita que esté rota para que aparezca la asimetría en masas, pero lo mínimo y necesario para preservar las características que solucionaban el problema de jerarquía. El lagrangiano efectivo del MSSM toma la forma:

\begin{equation}
	\mathcal{L}_{\text{MSSM}} = \mathcal{L}_{\text{SUSY}} + \mathcal{L}_{\text{soft}} 
	\label{eq:l_susy}
\end{equation}

donde $\mathcal{L}_{\text{SUSY}}$ contiene todas las interacciones de gauge y Yukawa y preserva la inavarianza frente a supersimetría, y $\mathcal{L}_{\text{soft}}$ viola supersimetría pero contiene solo términos de masa y parámetros de acoplamiento con dimensiones positivas de masa. La diferencia de masas que hay entre las partículas del SM y sus supercompañeros dependerá de la escala de masa más grande asociado al término \text{soft} ($m_{\text{soft}}$). Esta escala no puede ser indiscriminadamente grande ya que se perdería la solución al problema de jerarquía, ya que las correcciones a la masa del Higgs serían extremadamente grandes. Se puede estimar que $m_{\text{soft}}$, y por ende las masas de los supercompañeros más livianos, deben estar en la escala del TeV. Esto es una de las motivaciones más importantes en las búsquedas experimentales, principalmente en los experimentos del LHC-CERN.

% One might also wonder whether there is any good reason why all of the superpartners of the
% Standard Model particles should be heavy enough to have avoided discovery so far. There is. All of the
% particles in the MSSM that have been found so far, except the 125 GeV Higgs boson, have something
% in common; they would necessarily be massless in the absence of electroweak symmetry breaking. In
% particular, the masses of the W ± , Z 0 bosons and all quarks and leptons are equal to dimensionless
% coupling constants times the Higgs VEV ∼ 174 GeV, while the photon and gluon are required to be
% massless by electromagnetic and QCD gauge invariance. Conversely, all of the undiscovered particles
% in the MSSM have exactly the opposite property; each of them can have a Lagrangian mass term in the
% absence of electroweak symmetry breaking. For the squarks, sleptons, and Higgs scalars this follows
% from a general property of complex scalar fields that a mass term m 2 |φ| 2 is always allowed by all gauge
% symmetries. For the higgsinos and gauginos, it follows from the fact that they are fermions in a real
% representation of the gauge group. So, from the point of view of the MSSM, the discovery of the top
% quark in 1995 marked a quite natural milestone; the already-discovered particles are precisely those
% that had to be light, based on the principle of electroweak gauge symmetry. There is a single exception:
% it has long been known that at least one neutral Higgs scalar boson had to be lighter than about 135
% GeV if the minimal version of supersymmetry is correct, for reasons to be discussed in section 8.1. The
% 125 GeV Higgs boson discovered in 2012 is presumably this particle, and the fact that it was not much
% heavier can be counted as a successful prediction of supersymmetry.

En una teoría supsersimétrica renormalizable, la interaccion y las masas de todas las particulas estan determinadas solamente por las propiedades de sus transformaciones de gauge y por el superpotencial $W$. Del MSSM hasta ahora tenemos el grupo de gauge, las partículas del mismo y las propiedasdes de las transformaciones de gauge, resta describir en tonces el superpotencial que tomar la forma:

\begin{equation}
	W_{\text{MSSM}} = \bar{u}\textbf{y}_\textbf{u}QH_u - \bar{d}\textbf{y}_\textbf{d}QH_d - \bar{e}\textbf{y}_\textbf{e}LH_d + \mu H_u H_d
	\ref{eq:susy_potential}
\end{equation}

Los campos que aparecen son los mismos de la Tabla \ref{mssm_particles}, y las matrices 3x3 \textbf{y}_\textbf{u}, \textbf{y}_\textbf{d}, \textbf{y}_\textbf{e}, son los parámetros adimensionales del acoplamientos de Yukawa. Los índices para las transforamciones de gauge y familia de sabores fueron omitidos por practicidad. El último término con el parámetro $\mu$ es la versión supsersimétrica de la masa del Higgs del SM.

El superpotencial de la ecuación \ref{eq:susy_potential} 



