\thispagestyle{empty}

{\centering \bf \Large  Búsqueda de Supersimetría con producción de Higgs en el detector ATLAS (CERN-LHC)\\}

\vspace{2cm}

El Modelo Estándar es la teoría que describe las partículas elementales y sus interacciones, desarrollada en la década de los 70 y con grandes predicciones experimentales, tales como el descubrimiento del bosón de Higgs en el año 2012. A partir de su formulación, surgieron nuevas extensiones intentando solucionar diferentes problemáticas del mismo. Una de las extensiones mejor motivadas teóricamente es Supersimetría (SUSY), que introduce un conjunto de partículas nuevas que aún no han sido observadas. Este modelo, entre otras cosas, presenta un escenario favorable para la inclusión de la gravedad al Modelo Estándar, y a su vez, dichas nuevas partículas podrían ser candidatas tanto a materia oscura, como a neutrinos pesados. Esto ha convertido a SUSY en una de las teorías de mayor interés y el mayor objetivo en el ámbito de la física experimental de altas energías.

Esta tesis presenta una búsqueda de nueva física motivada por modelos de SUSY que predicen estados finales con fotones energéticos y aislados, jets y momento transverso faltante elevado. La misma fue realizada utilizando el conjunto de datos de colisiones $pp$, provisto por el Gran Colisionador de Hadrones del CERN, y recolectado por el detector ATLAS durante los años 2015 y 2018, correspondientes a una luminosidad integrada de $139\,\ifb$. En el presente trabajo se desarrollaron y realizaron búsquedas guiadas por modelos de producción fuerte de partículas supersimétricas, en las cuales no se observaron excesos por sobre las predicciones del Modelo Estándar, por lo que se establecieron límites en el número de eventos de nueva física, y adicionalmente límites en la producción de gluinos con masas de \magn{2.2}{TeV}. A su vez, se describe complementariamente el marco para la búsqueda de partículas supersimétricas con producción débil, cuyo estado final es similar al anterior descripto, y obteniéndose una posible sensibilidad de exclusión para partículas con masas de hasta \magn{1.2}{TeV}.
El estudio detallado de los datos requirió además la medida de la eficiencia de selección de los triggers de fotones del detector ATLAS, cuya técnica se describe en el presente trabajo, con resultados actualmente utilizados en todos los análisis con selección online de fotones.

