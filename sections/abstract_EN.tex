\thispagestyle{empty}

{\centering \bf \Large  Search for Supersymmetry with Higgs production with the ATLAS detector (CERN-LHC)\\}

\vspace{2cm}

The Standard Model is the theory that describes elementary particles and their interactions, developed in the 1970s and with great experimental predictions such as the discovery of the Higgs boson in 2012. From its formulation, new extensions arose trying to solve different problems. One of the best theoretically motivated extensions is Supersymmetry (SUSY), which introduces a new set particles that have not yet been observed. This model, among other things, presents a favorable scenario for the inclusion of gravity in the Standard Model, and in turn, these new particles could be candidates for both dark matter and heavy neutrinos. This has made SUSY one of the most interesting theories and the biggest target in the field of experimental high energy physics.

This thesis presents a search for new physics motivated by SUSY models that predict final states with energetic and isolated photons, jets and high missing transverse momentum. It was carried out using the $pp$ collision data set, provided by the CERN Large Hadron Collider, and collected by the ATLAS detector during the years 2015 and 2018, corresponding to an integrated luminosity of $139\,\ifb$ . In the present work, searches were developed and carried out guided by models of strong production of supersymmetric particles, in which no excesses were observed over the predictions of the Standard Model, for which limits were established on the number of new physics events, and additional limits on the production of gluinos with masses of \magn{2.2}{TeV}. At the same time, the framework for a search for supersymmetric particles with electroweak production and with a similar final state is described above, obtaining a possible exclusion sensitivity for particles with masses up to \magn{1.2}{TeV}. The detailed study of the data also required the measurement of the selection efficiency of the photon triggers of the ATLAS detector, whose technique is described in this work, with results currently used in all analyses with an online photon selection.


