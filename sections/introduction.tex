\chapter*{Introducción}
\addcontentsline{toc}{chapter}{Introducción}
\chaptermark{Introducción}

El Gran Colisionador de Hadrones (LHC) es el acelerador de partículas más grande y de mayor energía en todo el mundo, donde grupos de protones colisionan 40 millones de veces por segundo para producir colisiones protón-protón a 13 TeV de energía de centro de masa. Uno de los experimentos más importantes del LHC es ATLAS, un detector de uso general diseñado para realizar mediciones de precisión dentro del Modelo Estándar (SM) y de nuevos fenómenos asociados con nueva física que buscaban ser observados en la escala TeV. 

En el año 2012 las colaboraciones ATLAS y CMS publicaron resultados con el descubrimiento del bosón de Higgs, la partícula vinculada con el mecanismo de rompimiento espontaneo de simetría electrodébil, por el cual las partículas elementales adquieren masa. Sin embargo, todavía queda por determinar si es el boson de Higgs del modelo estándar o por ejemplo el más liviano de otros bosones de teorías más allá del modelo estándar, como es el caso de teorías supersimétricas. Para dar respuesta a este interrogante las colaboraciones tienen que medir con muy alta precisión, entre otras características, las distintas tasas de decaimiento a otras partículas y comparar los resultados con las predicciones.

Luego del descubrimiento del bosón de Higgs, son varios los interrogantes sin respuesta del 
SM, como por ejemplo, el patrón de las diferencias de masa de las partículas fundamentales y el problema de la jerarquía en la enorme diferencia de 17 órdenes de magnitud entre las dos escalas fundamentales de física: la escala electro-débil y la escala de Planck. Una de las ideas más intensamente investigadas desde el punto de vista teórico entre los modelos más allá del SM, es la ya mencionada supersimetría (SUSY). En su formulación mínima, SUSY predice que para cada partícula del SM existe un compañero cuyo spin difiere en $1/2$ y un sector de Higgs extendido con cinco bosones respectivos. La simetría propuesta entre bosones y fermiones estabiliza la masa de las partículas escalares, como es el caso del bosón de Higgs. Si las partículas propuestas conservan la paridad R (número cuántico propuesto por la teoría) entonces las partículas SUSY son siempre producidas de a pares y la más liviana (LSP) no puede decaer, con lo cual las LSP primordiales serían candidatos a formar la materia oscura, otro de los misterios para el cual el SM todavía no tiene respuesta. Las partículas supersimétricas pueden ser producidas en el LHC si su rango de masas está en la escala del TeV. La búsqueda de partículas SUSY en el LHC es entonces el objetivo más general del presente trabajo, en particular dentro del contexto del modelo General Gauge Mediated Symmetry Breaking (GMSB), en base a la cual se obtuvieron los limites más rigurosos en la masa de distintas partículas en estado finales con fotones, jet y energía transversa faltante en los canales de producción fuerte. El Run 2 del LHC tenía una mayor energía de centro de masa y luminosidad que su antecesor, permitiendo el acceso a secciones eficaces muy pequeñas, con la posibilidad de realizar búsquedas mucho más dedicadas en canales exclusivos. Esto brinda el marco apropiado para el desarrollo de búsquedas de supersimetría en el canales con producción débil, como se discute en también en este trabajo.

Las condiciones del Run 2 del LHC generaron nuevos desafíos para el detector ATLAS, en particular para el sistema de trigger del detector, el cual tiene que seleccionar eventos de interés físico, como por ejemplo para la búsqueda de supersimetría con fotones en estado final estudiados en este trabajo, sobre un enorme fondo de eventos. El calorímetro de ATLAS es el responsable de medir la energía de los fotones y electrones y también es utilizado para reducir la frecuencia de eventos aceptados a nivel del trigger calorimétrico. Nuevos criterios y sistemas se han implementado para calcular cantidades físicas en base a varios objetos de triggers, que al mismo tiempo de reducir la frecuencia permiten guardar los eventos de interés para su posterior análisis. Entre los resultados específicos del presente trabajo se estudiaron las prestaciones de los algoritmos de selección en base a los nuevos criterios en la toma de datos, resultados que fueron luego utilizados por toda la colaboración en todos los estudios que involucran la selección de fotones online que formen parte en distintos estados finales. 
