\chapter*{Introducción}
\addcontentsline{toc}{chapter}{Introducción}
\chaptermark{Introducción}

El Gran Colisionador de Hadrones (LHC) es el acelerador de partículas más grande y de mayor energía en todo el mundo, donde paquetes de protones colisionan 40 millones de veces por segundo para producir colisiones protón-protón. El mismo se encuentra en la frontera franco-suiza, aproximadamente a \magn{100}{m} bajo tierra, y consiste en un anillo de \magn{27}{km} de radio por donde circulan protones en direcciones opuestas. Los mismos colisionan en cuatro puntos estratégicos donde se encuentran los detectores principales: ATLAS, CMS, LHCb y ALICE.
Entre los años 2015 y 2018 se realizó la toma de datos denominada Run 2, en la cual los protones colisionaban con una energía de centro de masa de \magn{13}{TeV}, y lográndose recolectar una luminosidad total integrada $139\,\ifb$. En la actualidad el LHC se está preparando para comenzar con el Run 3, que colisionará protones con una energía de centro de masa de \magn{13.6}{TeV}, esperándose alcanzar una luminosidad total integrada de aproximadamente $300\,\ifb$.

Uno de los experimentos más importantes del LHC es ATLAS, un detector de uso general diseñado para realizar tanto mediciones de precisión dentro del Modelo Estándar (SM), como búsquedas de nuevos fenómenos asociados con física más allá del SM, que esperan ser observados en la escala TeV. El detector ATLAS se compone de distintos subdetectores que cumplen diferentes roles en la reconstrucción de las partículas producto de la colisión. El Detector Interno se encarga de medir las trazas de las partículas cargadas, los Calorímetros son los encargados de medir las deposiciones energéticas de fotones, electrones y diferentes hadrones, y finalmente el Espectrómetro de Muones permite medir la trayectoria de los muones. Intercalados entre ellos, se encuentra un poderoso sistema de imanes, que curva la trayectoria de las partículas cargadas. Por último, el detector ATLAS consta de un preciso Sistema de Trigger que filtra aquellos eventos de poco interés, reduciendo así la frecuencia de flujo de datos.


% Sin embargo, todavía queda por determinar si es el bosón de Higgs del modelo estándar o por ejemplo el más liviano de otros bosones de teorías más allá del modelo estándar, como es el caso de teorías supersimétricas. Para dar respuesta a este interrogante las colaboraciones tienen que medir con muy alta precisión, entre otras características, las distintas tasas de decaimiento a otras partículas y comparar los resultados con las predicciones.

En el año 2012 las colaboraciones ATLAS y CMS publicaron resultados con el descubrimiento del bosón de Higgs, la partícula vinculada con el mecanismo de rompimiento espontáneo de simetría electrodébil, por el cual las partículas elementales adquieren masa. Dicho descubrimiento valió un premio Nobel a los físicos Peter Higgs y François Englert, que postularon el mecanismo que lleva su nombre, y que era una pieza clave del SM. Desde su postulación hasta la fecha, el SM ha realizado fuertes predicciones que han sido verificadas experimentalmente, convirtiéndolo en una de las teorías más importantes de toda la física. Sin embargo, son varios los interrogantes sin respuesta del 
SM, como por ejemplo, el patrón de las diferencias de masa de las partículas fundamentales, y el problema de la jerarquía en la enorme diferencia de 17 órdenes de magnitud entre las dos escalas fundamentales de física: la escala electrodébil y la escala de Planck. Una de las ideas más intensamente investigadas desde el punto de vista teórico entre los modelos más allá del SM, es la Supersimetría (SUSY). En su formulación mínima, SUSY predice que para cada partícula del SM existe un compañero cuyo spin difiere en $1/2$, y un sector de Higgs extendido con cinco bosones respectivos. La nueva simetría propuesta entre bosones y fermiones estabiliza la masa de las partículas escalares, como es el caso del bosón de Higgs. Si las partículas propuestas conservan la paridad R (número cuántico propuesto por la teoría) entonces las partículas SUSY son siempre producidas de a pares y la más liviana (LSP) no puede decaer, con lo cual las LSP primordiales serían candidatas a formar la materia oscura, otro de los misterios para el cual el SM todavía no tiene respuesta. Estas nuevas partículas supersimétricas pueden ser producidas en el LHC si su rango de masas está en la escala del TeV. 

La búsqueda de partículas SUSY en el LHC es entonces el objetivo más general del presente trabajo, en particular dentro del contexto del modelo General Gauge Mediated Symmetry Breaking (GMSB), en base a la cual se obtuvieron los límites más rigurosos en las masas de distintas partículas, en estados finales con fotones, jet provenientes principalmente del decaimiento del bosón de Higgs, y energía transversa faltante, en los canales de producción fuerte. Debido a que el Run 2 del LHC operó a una mayor energía de centro de masa y luminosidad que su antecesor, se posibilitó el acceso a secciones eficaces menores, con la posibilidad de realizar búsquedas dedicadas en canales exclusivos. Esto brinda el marco apropiado para el desarrollo de búsquedas de supersimetría en canales con producción débil, como se describe también en este trabajo. 

Las nuevas condiciones del Run 2 del LHC, generaron nuevos desafíos para el detector ATLAS, en particular para el Sistema de Trigger, el cual tiene que seleccionar eventos de interés físico, como por ejemplo para la búsqueda de supersimetría con fotones en estado final estudiados en este trabajo, sobre un enorme fondo de eventos. Nuevos criterios y sistemas se implementaron para calcular cantidades físicas en base a varios objetos de triggers, que al mismo tiempo de reducir la frecuencia, permiten guardar los eventos de interés para su posterior análisis. Entre los resultados específicos del presente trabajo se estudiaron las prestaciones de los algoritmos de selección en base a los nuevos criterios en la toma de datos, resultados que fueron luego utilizados por toda la colaboración en todos los estudios que involucran la selección de fotones online que formen parte en distintos estados finales. 
